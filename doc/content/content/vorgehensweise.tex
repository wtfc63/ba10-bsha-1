\chapter{Vorgehensweise}
Für die Lösung des Problems gibt es schon einen Ansatz aus einer vorhergehenden Arbeit. Wir werden den vorgeschlagenen Ansaz aus dieser Arbeit in einem Prototypen umsetzen. Dieser Prototyp wird jedoch so aufgebaut, dass jeder Teil der Erkennung beliebig ausgetauscht werden kann. Damit können wir nicht nur eine Variante, sondern für jeden Schritt mehrere testen. Nach dem Vergleich der verschiedenen Varianten werden wir die besten auswählen und damit die Release-Version für Android erstellen.

Im Endeffekt werden wir zwei verschiedene Erkennungsvarianten testen: Variante A ist aus der vorhergehenden Arbeit übernommen und Variante B wird durch uns selbst erstellt. Der Vergleich basiert schlussendlich auf der Erkennungsrate für eine bestimmte Anzahl zufällig ausgewählter Zeichen.