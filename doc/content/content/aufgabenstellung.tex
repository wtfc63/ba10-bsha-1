\chapter{Aufgabenstellung}
Die Studenten realisieren eine Software in Android für das Erkennen von Handschrift, welche via Touchscreen eingegeben wird.

*Die Studierenden entwerfen eine Software, welche Touchscreen-Eingaben erkennt und auf dem Bildschirm wiedergibt
*Die Software soll sowohl einzelne Zeichen als auch mehrere zusammenhängende Zeichen („Schnürlischrift“) erkennen.
*Wenn der Finger angehoben wird, bedeutet das das Ende des Schriftzugs
*Die Erkennung verfolgt den Ansatz, dass verschiedene Zeichen sich aus Mikrogesten zusammensetzen. Diese Mikrogesten kommen in mehreren Zeichen vor.
*Die Applikation wird derart realisiert, dass die übrigen Funktionen des Mobilgerätes nicht behindert werden. Sie verfügt über die nötigen Bedienelemente (z.B. Bildschirm löschen etc.)