\chapter{Diskussion}
Zum Abschluss können wir sagen, dass die Idee der Mikrogesten zum erkennen von Handschriften definitiv umsetzbar ist. Jedoch sind wir nicht zu einer perfekten Lösung gelangt. Im Prinzip ist die Umsetzung zwar funktionsfähig, jedoch besteht bei der Erkennung der Mikrogesten noch etwas Verbesserunsbedarf und für die Erstellung des Graphen sind noch bessere Tools nötig um die Lösung für den täglichen Gebrauch praktikabel zu machen.

Dank des modularen Aufbaus unserer Applikation ist es aber sehr einfach die Erkennung mit neuen Algorithmen und Ansätzen zu erweitern oder bestehende Algorithmen zu verbessern.

Die Entwicklung mit Android hat sich als sehr einfach erwiesen: Von Google werden gute Entwicklungs-Werkzeuge bereit gestellt, die auch auf Mac, Windows und Linux verfügbar sind. Die API von Android ist gut dokumentiert und der flexible Aufbau des Betriebssystems erlaubte uns eine einfache Integration unserer Applikation. 