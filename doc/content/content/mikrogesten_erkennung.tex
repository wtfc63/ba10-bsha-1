\chapter{Mirkogesten-Erkennung}

Nach der Sichtung der vorgegangen Projektarbeit haben wir uns entschieden, dass zwei Ansätze zur Erkennung der einzelnen Mikrogesten aus der Eingabe-Menge an Punkten im Prototyp ausprobiert und evaluiert werden sollten. Zum einen sollte die in der Projektarbeit favorisierte Erkennung durch die Änderung der Krümmung zwischen den einzelen Punkten getestet werden, zum Anderen ein Ansatz ausprobiert werden, der auf einer auf dem vorausgehenden Punkt basierenden Voraussage des jeweils als nächstes folgenden Punktes aufbaut.

Der Krümmungs-Ansatz wurde in der vorgängigen Projektarbeit neben einem auf der Änderung der Winkel zwischen Punkten basierenden Ansatz zur Mikrogesten-Erkennung evaluiert und am Ende von den Autoren der Arbeit favorisiert. Dadurch war klar, dass wir diese Art der Erkennung ebenfalls für diese Arbeit evaluieren sollten.

Neben einer Erkennung durch die Krümmungs- und die Winkel-Änderung zwischen Punkten sollte unserer Meinung nach auch eine Erkennung durch die Voraussage von folgenden Punkten möglich sein und wurde daher von uns ebenfalls implementiert und evaluiert.

Als Resultat der Evaluation der beiden genannten Verfahren haben wir uns dazu entschieden, den Ansatz der Erkennung durch die Krümmungs-Änderung weiter zu verfolgen. Während der Evaluation stellte sich heraus dass das auf Voraussagen basierende Verfahren nicht tolerant genug auf kleine Änderungen innerhalb von Mirkogesten reagiert und daher dazu tendiert, diese aufzuteilen. Das Krümmungs-Verfahren wurde von uns als toleranter und daher besser für den mobilen Einsatz geeignet befunden.

\section{Bewertungs-Kriterien}

Kriterium:	Erkennungszuverlässigkeit
Beschreibung:	Um eine verlässliche Erkennung der Mikrogesten zu ermöglichen, muss ein Ansatz natürlich gewährleisten, dass er die Auftrennung der Mirkogesten zuverlässig vornimmt.
Wichtigkeit:	Substantiell

Kriterium:	Toleranz
Beschreibung:	Obwohl die Erkennung natürlich zuverlässig sein sollte, muss aber auch eine gewisse Toleranz für nicht ideale Eingaben gewährt werden. Da die Handschriften-Erkennung auf Menschen und einen Einsatz im mobilen Umfeld ausgerichtet werden soll, sind suboptimale Eingabefolgen durchaus zu erwarten und die Erkennung der Mikrogesten sollte daher einigermassen tolerant auf solche reagieren. Dies stellt offensichtlich ein Zielkonflikt zum Zuverlässigkeits-Kriterium dar und es muss daher ein guter Mittelweg zwischen diesen beiden Kriterien gefunden werden.
Wichtigkeit:	Substantiell

Kriterium:	Performanz
Beschreibung:	Auf einem mobilen System ist der schonende Umgang mit den Systemressourcen besonders wichtig und das Verfahren zur Erkennung der Mikrogesten sollte daher möglichst wenig Rechenleistung brauchen. Da es allerdings nur jeweils nach Benutzereingaben durchgeführt werden muss und eine Ausführung im Hintergrund möglich sein sollte, könnten gewisse Abstriche in diesem Bereich unter Umständen in Kauf genommen werden.
Wichtigkeit:	Sekundär
%================================================

\section{Ansatz: Krümmungs-Änderung}

%TODO

\subsection{Verfahren}

%TODO - Flussdiagramm Krümmungs-Verfahren

\subsection{Bewertung}

%TODO
%================================================

\section{Ansatz: Punkte-Voraussage}

Grundlage des auf Voraussage des jeweils nächsten Punktes basierenden Ansatzes sollte die Annahme sein, dass man den als nächstes auf einen Punkt einer gleichmässig gekrümmten Kurve folgenden Punkt durch das Spiegeln des jeweils dem Ausgangspunkt vorangehenden Punktes an der Achse des Ausgangspunktes vorherbestimmen können müsste. Weicht der tatsächlich eingegebene Folgepunkt zu sehr vom vorausgesagten Folgepunkt ab, sollte nun angenommen werden, dass dieser Punkt nicht mehr zur aktuellen Mikrogeste gehören und als Ausgangspunkt einer neuen Mikrogeste genommen werden soll.

\subsection{Verfahren}

%TODO - Flussdiagramm Prediction-Verfahren

\subsection{Bewertung}

Die grundsätzliche Trennung der einzelnen Mikrogesten druch das auf Voraussagen basierende Verfahren wurde von uns als durchaus zufriedenstellend bewertet. Das Verfahren ermöglicht die Auftrennung der Eingabepunkte in verschiedene Mikrogesten. Allerdings erkennt das Verfahren unserer Meinung nach leider zu viele einzelne Mikrogesten und ist zu wenig tolerant gegenüber kleineren Abweichungen. Schon kleinere Änderungen in der Linienführung führen dabei zu einer Auftrennung von ansonsten durchaus zusammenhängenden Gesten.

Auch weist das Verfahren Probleme mit ungleichen Punktedichten auf einem Pfad auf. Dies tritt vor allem auf, wenn die Eingabegeschwindigkeit variiert, etwa wenn nach einer Richtungsänderung oder am Anfang einer Geste die Eingabe beschleunigt wird und die späteren Eingabepunkte weiter auseinander liegen als die ersten. Gerade bei unregelmässigen Beschleunigungen kommt es somit häufig vor, dass die ersten Eingabepunkte relativ nahe beieinander liegen, der Abstand zu späteren Punkten dann aber recht gross wird. Die vorausgesagten Punkte mögen dann zwar durchaus auf dem Pfad der Geste liegen, die Eingabepunkte folgend dann allerdingst erst in grösserem Abstand. Diesem Verhalten könnte möglicherweise mit Anpassungen an der Berechnung der vorausgesagten Punkte begegnet werden. Man könnte etwa anstatt eines konkreten Punktes eine Richtung voraussagen. Auch ein normierter Abstand zwischen den Eingabepunkten, wie ihn das Bezier-Glättungsverfahren erzeugt, welches als mögliche Optimierung in \ref{sec:Glaettung} besprochen wird. Eine Glättung der Eingangspunkte sollte auch die Toleranz des voraussagenden Ansatzes verbessern können.

Weiter haben wir mit diesem Verfahren Probleme mit zu nahe beieinander liegenden Eingabepunkten festgestellt. Ist die Distanz von einem Punkt zum nächsten kleiner als die Länge des Toleranz-Wertes, fällt der Punkt in jedem Fall innerhalb der Toleranz und es kann keine fundierte Aussage über mehr darüber gemacht werden, ob der Punkt zur selben Mirkogeste gehören soll oder nicht. Daher werden nur Punkte, deren Abstand zum Ausgangspunkt kleiner als die zweifache Länge des Toleranz-Wertes sind, beachtet. Andere Punkte werden als zur aktuellen Mikrogeste gehörig angenommen und zum neuen Ausgangspunkt. Dies führt wiederum zu Problemen mit sehr engen Kurven, wie sie etwa bei Richtungswechseln und sehr engwinkligen Spitzen, da diese eng beieinander liegende Eingabepunkte erzeugen, deren Abstand häufig unterhalb die Toleranzgrenze fällt. Somit wird häufig noch der erste Punkt nach einer solchen Spitze zur vorherigen Mikrogeste zugeordnet.

Diesen beiden Probleme, welche beide das als substantiell gewertete Bewertungskriterium der Toleranz beinträchtigen, steht steht nur eine gute Performanz durch die relativ simplen Berechnungen, die das Voraussage-Verfahren benötigt, gegenüber. Da das Toleranz-Kriterium von uns aber als signifikant wichtiger eingestuft wurde und vom Krümmungs-Verfahren weit besser erfüllt wird, mussten wir uns am Ende für das auf Krümmungs-Änderungen basierende Verfahren festlegen.
%================================================

\section{Erkennung der Mikrogesten-Art}

%TODO
%================================================

\section{Erkennung der Ausrichtung einer Mikrogeste}

%TODO
%================================================

\section{Allgemeine Optimierungen}

%TODO

\subsection{Zusammenfügen von minimalen Mikrogesten}

%TODO

\subsection{Glättung der Eingangspunkte}\label{sec:Glaettung}

%TODO - Verschiedene Verfahren, Optimierungen
%================================================

