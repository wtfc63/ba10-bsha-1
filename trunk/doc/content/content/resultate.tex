\chapter{Resultate}
% - bessere Algorithmen zur MG-Erkennung
% - Character-Erkennung mit Graph
% - zwei MG-Varianten mit passenden Graphen
% - Import, Export, Visualisierung Graph
% - Software zum testen der Mikrogesten-Erkennung
% - Ersatz für das Keyboard im Android Betriebssystem
% - ???
% - Profit.

\begin{enumerate}
\item \textbf{Entwicklung einer Input-Method und Handschrifterkennungs-Service für Android} \\ 
Die erstellte Input-Method kann als Ersatz für das Standard-Keyboard von Android verwendet werden. Für die Erkennung der Buchstaben greift sie auf den Service zu, der die gesamte Erkennungs-Logik enthält und von der Input-Method unabhängig läuft.
Der Service ist komplett modular aufgebaut, was es erlaubt auf sehr einfache Weise die Logik zur Buchstabenerkennung zu ändern.

\item \textbf{Zwei Varianten für die Aufteilung von Pfaden in Mikrogesten} \\
Wir haben zwei verschiedene Varianten von Mikrogesten implementiert und verglichen. 

\item \textbf{Buchstaben-Erkennung basierend auf einem Graphen} \\
Für jede der Mikrogesten-Varianten haben wir einen Graph erstellt, welcher mit Hilfe der erkannten Mikrogesten einen Buchstaben auswählt. Für den Graph haben wir ausserdem noch Import-, Export- und automatische Visualisierungs-Klassen eingebaut.

\item \textbf{Neue und verbesserte Algorithmen für die Mikrogesten-Erkennung} \\
Wir haben einen neuen Ansatz entwickelt, um Mikrogesten zu erkennen. Statt in einem Schritt alle verschiedenen Mikrogesten zu erkennen und bestimmen, wird der Pfad über mehrere Stufen verarbeitet. 

\end{enumerate}