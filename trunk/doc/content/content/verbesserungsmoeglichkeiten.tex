\chapter{Verbesserungsmöglichkeiten}
Unsere Lösung für das Problem besitzt noch einiges an Verbesserungspotential. In diesem Kapitel werden wir einige Ideen aufzeigen, die wir aus Zeitgründen nicht mehr umsetzen konnten.

\section{Mikrogesten-Erkennung}


\section{Zeichen-Erkennung}
\subsection{Neuronales Netz}
Als Alternative zum Graphen bietet isch ein neuronales Netz an. Es gibt bereits speziell auf Handschrifterkennung ausgerichtete neuronale Netze zum Beispiel von Microsoft. Das spezielle an diesen Netzen ist, dass nicht alle Eingangsdaten auf einmal eingegeben werden müssen, sondern sequentiell ins Netz gespeist werden können. Dies passt auch sehr gut auf unsere Architektur, da der Graph die Mikrogesten ebenfalls sequentiell verarbeitet.

Ein weiterer Vorteil der neuronalen Netze sind die vielfältigen Trainingsmöglichkeiten. Das Netz könnte dann einfach zuerst vom Benutzer trainiert werden und ist somit schon auf seine Handschrift eingestellt. Die Erkennungsrate der Mikrogesten muss dadurch nicht mehr perfekt sein, da ein neuronales Netz im Gegensatz zum Graphen unscharf arbeitet. 

\subsection{Graph}
Bei der Graph-Variante der Zeichen-Erkennung gibt es auch Verbesserungspotential: Vor allem für die Erstellung des Graphen sind bessere Tools nötig. Wir haben momentan eine Datenstruktur für das speichern des Graphen und ein Tool zur Visualisierung erstellt. Ein nächster Schritt wäre die Erweiterung des Visualisierungs-Tools, so dass dieses Graphen editieren und speichern kann. Theoretisch müssten mit dem Graphen auch sehr gute Erkennungsraten möglich sein, jedoch nur mit erheblichem Zeitaufwand. Ein Editier-Werkzeug könnte diesen Vorgang stark beschleunigen.

\subsection{Wörterbuch}
In unserer Software-Architektur ist bereits vorgesehen, nach der Zeichenerkennung noch eine Verifizierung des Buchstabens durchzuführen. Dazu wäre eine Überprüfung mittels Wörterbuch eine gute Lösung. Das Android-Betriebssystem enthält bereits ein solches Wörterbuch für die normale Bildschirmtastatur. Man könnte versuchen, auf dieses Wörterbuch zuzugreifen und es im Service zu integrieren. Dies könnte zum Beispiel in Form einer Nachverarbeitungs-Strategie umgesetzt werden.

\section{Erkennungs-Service für Android}
Obwohl die grundsätzliche Infrastruktur dazu überwiegend vorhanden wäre, werden momentan im Service nur ganze Blöcke an Eingabe-Punkten behandelt und keine Zwischenstände verarbeitet. Dies wäre allerdings für eine flüssiges Benutzer-Erlebnis von Vorteil und sollte daher in Betracht gezogen werden.

\section{Benutzer-Oberfläche}
Die momentane Benutzeroberfläche ist sehr funktional ausgerichtet und müsste für Endbenutzer wohl rechts stark ausgebaut werden.

Zum Einen wird im Moment keine Benutzer-Oberfläche zu Konfiguration der Anwendung bereit gestellt. Eine solche wäre allerdings wünschenswert, damit der Benutzer auch das Verhalten der Erkennung zur Laufzeit konfigurieren kann. Auch hier existiert bereits der überwiegende Teil des benötigten Unterbaus, etwa in der Form der Argumente für die Strategien, diese Funktionalität müsste aber dem Benutzer noch zugänglich gemacht werden.

Auch die Möglichkeiten der Eingabe-Methode sind zur Zeit relativ beschränkt. Neben der eigentlichen Zeichen-Eingabe werden nur sehr rudimentäre Funktionen zum Löschen des letzten Zeichens und für das Einfügen von Leerzeichen zur Verfügung gestellt. Das Android System sieht für Eingabe-Methoden etwa Mechanismen vor um basierend auf den bisher eingegebenen Zeichen die daraus wahrscheinlichen Wörter als Kandidaten anzubieten. Unsere Eingabe-Methode besitzt auch hier die grundsätzliche Fähigkeit, dies umzusetzen, diese wurden allerdings nicht umgesetzt. Auch die Möglichkeit, Eingaben elegant fortzusetzen wenn sie den Bildschirmrand erreichen, wäre sehr zu begrüssen. So wäre es etwa vorstellbar, dass eine Zone am rechten Bildschirmrand definiert und ausgewiesen wird, in der Eingaben nach Absetzen des Fingers noch nicht als abgeschlossen betrachtet werden. Diese können dann am linken Rand fortgesetzt werden, gelten aber immer noch als eine geschlossene Einheit. Ohne einen solchen Mechanismus ist das Eingeben von ganzen Wörtern am Stück vor allem im Porträt-Modus nur sehr eingeschränkt möglich. Nur wenn das Endgerät quer gehalten wird steht normalerweise genügen Platz zu Eingabe von ganzen Wörtern zur Verfügung.

Auch die grafische Gestaltung wäre natürlich noch ausbaufähig. Vor allem eventuelle Endbenutzer scheinen tendenziell etwas ``Eye Candy'' durchaus nicht abgeneigt zu sein.
