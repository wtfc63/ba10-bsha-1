\chapter{Aufbau Report}
Der Report ist im Prinzip in drei Teile aufgeteilt:
\begin{itemize}
\item Theoretischer Hintergrund
\item Lösung / Umsetzung
\item Diskussion
\end{itemize}

Der Theoretische Hintergrund beschreibt dabei die Idee, wie die Handschrift erkannt werden soll. 

Die Lösung zeigt die Umsetzung der theoretischen Ideen in einem Java Programm. In diesem Teil werden vor allem die Algorithmen vorgestellt, die wir verwenden.

Die Diskussion enthält den Vergleich der verschiedenen Ansätzen. Ausserdem werden dort noch einige Vorschläge gemacht, wie man die von uns erstellte Lösung noch verbessern kann.